\chapter{\ifenglish Introduction\else บทนำ\fi}

\section{\ifenglish Project rationale\else ที่มาของโครงงาน\fi}
จากปัญหาที่ได้พบจากการสอบถามกับทางสำนักหอสมุดและคนรอบตัวที่ใช้บริการของสำนักหอสมุดเป็นประจำ พบว่าในแต่ละวันมีนักศึกษามหาวิทยาลัยเชียงใหม่จำนวนมากที่มาใช้หอสมุดโดยเฉพาะในช่วงสอบ แต่อาจไม่สามารถหาที่นั่งว่างได้ง่าย ๆ 
เพราะยังไม่มีระบบการแจ้งเตือนที่ช่วยบอกล่วงหน้าว่ามีที่นั่งว่างในหอสมุดหรือไม่ นักศึกษาจึงจำเป็นต้องเดินทางมาที่หอสมุดแล้วเดินหาที่ว่างด้วยตนเอง ทำให้เสียเวลาในการค้นหาและยังทำให้สำนักหอสมุดมีความแออัดโดยไม่จำเป็นอีกด้วย
ดังนั้น โครงงานนี้จึงช่วยแก้ไขปัญหาที่กล่าวมาโดยการพัฒนาเว็บแอปพลิเคชันที่ช่วยให้ผู้ใช้งานสามารถรู้ตำแหน่งที่นั่งว่างในสำนักหอสมุดได้ก่อนเข้าใช้บริการ ทำให้ผู้ใช้สามารถวางแผนการหาสถานที่อ่านหนังสือได้ง่ายขึ้น 
และไม่เสียเวลาไปกับการหาที่นั่งที่ว่างอยู่ด้วย
\section{\ifenglish Objectives\else วัตถุประสงค์ของโครงงาน\fi}
\begin{enumerate}
    \item เพื่อให้ผู้ใช้งานหาที่นั่งได้ง่ายขึ้น
    \item เพื่อให้ผู้ใช้งานสามารถทราบล่วงหน้าว่ายังมีที่นั่งว่างเหลืออยู่หรือไม่ 
    \item เพื่อให้สำนักหอสมุดสามารถบันทึกสถิติแล้วนำไปปรับปรุงหอสมุดให้ดีขึ้น
\end{enumerate}

\section{\ifenglish Project scope\else ขอบเขตของโครงงาน\fi}

\subsection{\ifenglish Hardware scope\else ขอบเขตด้านฮาร์ดแวร์\fi}
โมเดล จะมีการทำงานภายในสำนักหอสมุด มหาวิทยาลัยเชียงใหม่
\subsection{\ifenglish Software scope\else ขอบเขตด้านซอฟต์แวร์\fi}

\section{\ifenglish Expected outcomes\else ประโยชน์ที่ได้รับ\fi}

\section{\ifenglish Technology and tools\else เทคโนโลยีและเครื่องมือที่ใช้\fi}

\subsection{\ifenglish Hardware technology\else เทคโนโลยีด้านฮาร์ดแวร์\fi}
\begin{itemize}
    \item ESP32-CAM with OV2640
\end{itemize}
\subsection{\ifenglish Software technology\else เทคโนโลยีด้านซอฟต์แวร์\fi}
\begin{itemize}
    \item figma
    \item Python
    \item MongoDB
    \item React
    \item Node.js
    \item Visual Studio Code    
\end{itemize}
\section{\ifenglish Project plan\else แผนการดำเนินงาน\fi}

\begin{plan}{11}{2022}{3}{2023}
    \planitem{11}{2022}{12}{2022}{ศึกษาค้นคว้าข้อมูลในการทำโครงงาน}
    \planitem{12}{2020}{1}{2023}{ทดสอบการทำงานของ model}
    \planitem{2}{2021}{2}{2021}{ทดสอบระบบของ application}
    \planitem{12}{2019}{1}{2022}{เขียนรายงานและนำเสนอ}
\end{plan}

\section{\ifenglish Roles and responsibilities\else บทบาทและความรับผิดชอบ\fi}
\begin{itemize}
\item ส่วนที่ทำงานร่วมกัน ได้แก่ การวางแผนการทำงาน การค้นหาข้อมูล และเครื่องมือ
\item ส่วนที่รับผิดชอบโดย นางสาว ชวัลลักษณ์ แก้วมูล ได้แก่ การพัฒนาและประยุกต์ใช้ model และ library ที่มีอยู่เกี่ยวกับ object detection และ Database
\item ส่วนที่รับผิดชอบโดย นาย ธนดล ตระกูลขยัน ได้แก่ การออกแบบ UI/UX ของระบบ และการพัฒนา Web application
\end{itemize}
\section{\ifenglish%
Impacts of this project on society, health, safety, legal, and cultural issues
\else%
ผลกระทบด้านสังคม สุขภาพ ความปลอดภัย กฎหมาย และวัฒนธรรม
\fi}
โครงงานเว็บแอปพลิเคชันสำหรับการค้นหาตำแหน่งที่นั่งที่ยังว่างอยู่ภายในสำนักหอสมุดมหาวิทยาลัยเชียงใหม่ เป็นโครงงานที่
นำการวิเคราะห์ของ Machine Learning เข้ามาช่วยในการ ซึ่งจะช่วยประกอบการ
ตัดสินใจให้กับผู้ที่ โดยโครงงานได้คำนึงถึงกฎหมาย PDPA หรือ พ.ร.บ. คุ้มครองข้อมูลส่วนบุคคล ข้อมูลที่
โครงงานได้นำมาใช้วิเคราะห์นั้น 
ไม่ระบุตัวตน ซึ่งทำให้ไม่มีความกังวลที่ข้อมูลส่วนตัวจะ
ถูกเปิดเผย นอกจากนั้นโครงงานนี้ยังเป็น
