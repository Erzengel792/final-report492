\chapter{\ifenglish Introduction\else บทนำ\fi}

\section{\ifenglish Project rationale\else ที่มาของโครงงาน\fi}
จากปัญหาที่ได้พบจากการสอบถามกับทางสำนักหอสมุดและคนรอบตัวที่ใช้บริการของสำนักหอสมุดเป็นประจำ พบว่าในแต่ละวันมีนักศึกษามหาวิทยาลัยเชียงใหม่จำนวนมากที่มาใช้หอสมุดโดยเฉพาะในช่วงสอบ แต่อาจไม่สามารถหาที่นั่งว่างได้ง่าย ๆ 
เพราะยังไม่มีระบบการแจ้งเตือนที่ช่วยบอกล่วงหน้าว่ามีที่นั่งว่างในหอสมุดหรือไม่ นักศึกษาจึงจำเป็นต้องเดินทางมาที่หอสมุดแล้วเดินหาที่ว่างด้วยตนเอง ทำให้เสียเวลาในการค้นหาและยังทำให้สำนักหอสมุดมีความแออัดโดยไม่จำเป็นอีกด้วย
ดังนั้น โครงงานนี้จึงช่วยแก้ไขปัญหาที่กล่าวมาโดยการพัฒนาเว็บแอปพลิเคชันที่ช่วยให้ผู้ใช้งานสามารถรู้ตำแหน่งที่นั่งว่างในสำนักหอสมุดได้ก่อนเข้าใช้บริการ ทำให้ผู้ใช้สามารถวางแผนการหาสถานที่อ่านหนังสือได้ง่ายขึ้น 
และไม่เสียเวลาไปกับการหาที่นั่งที่ว่างอยู่ด้วย
\section{\ifenglish Objectives\else วัตถุประสงค์ของโครงงาน\fi}
\begin{enumerate}
    \item เพื่อให้ผู้ใช้งานหาที่นั่งได้ง่ายขึ้น
    \item เพื่อให้ผู้ใช้งานสามารถทราบล่วงหน้าว่ายังมีที่นั่งว่างเหลืออยู่หรือไม่ 
    \item เพื่อให้สำนักหอสมุดสามารถบันทึกสถิติแล้วนำไปปรับปรุงหอสมุดให้ดีขึ้น
\end{enumerate}

\section{\ifenglish Project scope\else ขอบเขตของโครงงาน\fi}

\subsection{\ifenglish Hardware scope\else ขอบเขตด้านฮาร์ดแวร์\fi}
\begin{itemize}
    \item การใช้งาน ESP32 with OV2640 และติดตั้งในหอสมุดมหาวิทยาลัยเชียงใหม่ สำหรับ Object Detection
    \item การเชื่อมต่อ ESP32 กับ PC เพื่อส่งภาพจากกล้องมาที่ PC เพื่อประมวลผล
\end{itemize}
\subsection{\ifenglish Software scope\else ขอบเขตด้านซอฟต์แวร์\fi}
\begin{itemize}
    \item การใช้งาน MongoDB เพื่อเก็บข้อมูลเกี่ยวกับที่นั่งที่ว่างอยู่ในหอสมุด
    \item การใช้งาน OpenCV ร่วมกับ TensorFlow model เพื่อทำ Object Detection และนับจำนวนคนที่อยู่ในภาพ
    \item การพัฒนา Web application สำหรับเข้าถึงข้อมูลที่นั่งที่ว่างอยู่
\end{itemize}

\section{\ifenglish Expected outcomes\else ประโยชน์ที่ได้รับ\fi}
\begin{enumerate}
   \item นักศึกษาไม่ต้องเสียเวลาในการเดินวนหาที่นั่งว่างอยู่ในแต่ละชั้นของหอสมุด
   \item นักศึกษาสามารถใช้เวลาที่มีได้อย่างมีประสิทธิภาพมากยิ่งขึ้นในการอ่านหนังสือหรือทำงานในหอสมุด
   \item หอสมุดสามารถจัดการและปรับปรุงที่นั่งได้อย่างมีประสิทธิภาพ 
\end{enumerate}
\section{\ifenglish Technology and tools\else เทคโนโลยีและเครื่องมือที่ใช้\fi}

\subsection{\ifenglish Hardware technology\else เทคโนโลยีด้านฮาร์ดแวร์\fi}
\begin{itemize}
    \item ESP32-CAM with OV2640
    \item Acer Nitro 5
    \item Acer Aspire 7
\end{itemize}
\subsection{\ifenglish Software technology\else เทคโนโลยีด้านซอฟต์แวร์\fi}
\begin{itemize}
    \item Figma
    \item Python
    \item MongoDB
    \item React
    \item Node.js
    \item Visual Studio Code
    \item OpenCV
    \item Tensorflow
    \item Arduino IDE    
\end{itemize}
\section{\ifenglish Project plan\else แผนการดำเนินงาน\fi}

\begin{plan}{11}{2022}{10}{2023}
    \planitem{11}{2022}{12}{2022}{ศึกษาค้นคว้าข้อมูลในการทำโครงงาน}
    \planitem{12}{2022}{2}{2023}{ทดสอบการทำงานของ model ที่เลือกใช้}
    \planitem{2}{2023}{3}{2023}{เขียนรายงานและเตรียมนำเสนอ}
    \planitem{4}{2023}{6}{2023}{ออกแบบ Web Application}
    \planitem{6}{2023}{7}{2023}{ทดสอบการทำงานของ Web Application และติดตั้งอุปกรณ์}
    \planitem{8}{2023}{9}{2023}{ปรับปรุงและพัฒนาโครงงานให้ดีขึ้น}
    \planitem{9}{2023}{10}{2023}{สรุปผล ทำรายงาน และเตรียมนำเสนอ}
\end{plan}

\section{\ifenglish Roles and responsibilities\else บทบาทและความรับผิดชอบ\fi}
\begin{itemize}
\item ส่วนที่ทำงานร่วมกัน ได้แก่ การวางแผนการทำงาน การค้นหาข้อมูล และเครื่องมือ
\item ส่วนที่รับผิดชอบโดย นางสาว ชวัลลักษณ์ แก้วมูล ได้แก่ การพัฒนาและประยุกต์ใช้ model และ library ที่มีอยู่เกี่ยวกับ object detection และ Database
\item ส่วนที่รับผิดชอบโดย นาย ธนดล ตระกูลขยัน ได้แก่ การออกแบบ UI/UX ของระบบ และการพัฒนา Web application
\end{itemize}
\section{\ifenglish%
Impacts of this project on society, health, safety, legal, and cultural issues
\else%
ผลกระทบด้านสังคม สุขภาพ ความปลอดภัย กฎหมาย และวัฒนธรรม
\fi}
โครงงานเว็บแอปพลิเคชันสำหรับการค้นหาตำแหน่งที่นั่งที่ยังว่างอยู่ภายในสำนักหอสมุดมหาวิทยาลัยเชียงใหม่ เป็นโครงงานที่
นำการวิเคราะห์จากการทำ Object detection เพื่อช่วยอำนวยความสะดวกแก่ผู้ที่จะเข้ามาใช้งานพื้นที่ในสำนักหอสมุดมหาวิทยาลัยเชียงใหม่โดยเฉพาะอย่างยิ่งในช่วงสอบกลางภาคและปลายภาค
ในการวางแผนที่จะมาใช้บริการสำนักหอสมุดเป็นพื้นที่ในการอ่านหนังสือว่ามีที่นั่งเพียงพอในช่วงเวลาที่ต้องการมาเข้าใช้ และโครงงานได้คำนึงถึงกฎหมาย PDPA หรือ พ.ร.บ. คุ้มครองข้อมูลส่วนบุคคล 
โดยข้อมูลที่โครงงานได้นำมาใช้วิเคราะห์นั้นเป็นภาพจากมุมสูงและใช้การตรวจจับรูปร่างของคน อีกทั้งเมื่อทำการวิเคราะห์ภาพและเก็บข้อมูลจำนวนคนเสร็จก็จะใส่กล่องสี่เหลี่ยมทึบทับภาพคนแล้วบันทึกภาพนั้นทับกับภาพเดิมอีกที จึงไม่มีการระบุตัวตนของแต่ละบุคคลอย่างแน่นอน
ซึ่งทำให้ไม่มีความกังวลที่ข้อมูลส่วนตัวจะถูกเปิดเผย
