\chapter{\ifenglish Background Knowledge and Theory\else ทฤษฎีที่เกี่ยวข้อง\fi}

ในบทนี้ก็จะเกี่ยวกับการอธิบายถึงสิ่งที่เกี่ยวข้องกับโครงงาน เพื่อให้ผู้อ่านเข้าใจเนื้อหาในบทถัดๆ ไปได้ง่ายขึ้น

\section{ด้าน Backend}

\subsection{Computer Vision}
Computer Vision เป็นกลุ่มของเทคโนโลยีและเทคนิคการประมวลผลภาพและวิดีโอด้วยคอมพิวเตอร์ โดยการนำเอาสัญญาณภาพและวิดีโอจากกล้องหรือเซ็นเซอร์ไปประมวลผลด้วยโปรแกรมคอมพิวเตอร์เพื่อให้เข้าใจและวิเคราะห์ภาพได้อย่างถูกต้องและมีประสิทธิภาพ
\enskip
โดย Computer Vision สามารถใช้งานได้ในหลายแขนงเช่น การตรวจจับวัตถุ (object detection) การติดตามวัตถุ (object tracking) การจำแนกวัตถุ (object recognition) การแยกแยะสิ่งของในภาพ (image segmentation) การประมวลผลภาพแบบสัมพันธ์ (image processing) และอื่นๆ\cite{Vision}

\subsection{Machine Learning}
Machine Learning คือการเรียนรู้และวิเคราะห์ข้อมูลด้วยอัลกอริทึมเพื่อทำนายและการตัดสินใจโดยไม่ต้องระบุมาก่อนว่าคำตอบจะเป็นอย่างไร
เช่น การจำแนกภาพ การสแกนเอกสาร การจัดกลุ่มหรือแนะนำสินค้า ซึ่งจะมีการใช้ข้อมูลที่มีปริมาณมากและมีความซับซ้อน 
การเรียนรู้ของ Machine Learning จะเป็นการให้คอมพิวเตอร์เรียนรู้จากข้อมูล เพื่อทำนายผลลัพธ์ที่ถูกต้องที่สุดเท่าที่เป็นไปได้ สามารถแบ่งออกเป็น 3 ประเภท ได้แก่
\begin{enumerate}
  \item Supervised Learning: เป็นการเรียนรู้ด้วยข้อมูลที่มีการระบุผลลัพธ์ให้แล้ว เช่น การจำแนกภาพว่าเป็นแมวหรือหมา
  \item Unsupervised Learning: เป็นการเรียนรู้โดยไม่มีการระบุผลลัพธ์ให้แล้ว แต่จะต้องหาลักษณะหรือลำดับที่เป็นลักษณะเด่นเพื่อแบ่งกลุ่มข้อมูล เช่น การจัดกลุ่มลูกค้าที่มีพฤติกรรมการซื้อสินค้าคล้ายกัน
  \item Reinforcement Learning: เป็นการเรียนรู้โดยมีการให้รางวัลหรือลบคะแนนเมื่อทำกิจกรรมใดๆ เช่น การเล่นเกม
\end{enumerate}
\cite{Machine}
\subsection{NoSQL Database}
NoSQL ย่อมาจาก "Not Only SQL" ซึ่งเป็นระบบฐานข้อมูลที่ไม่ใช้ภาษา SQL ในการจัดการข้อมูล ต่างจากระบบฐานข้อมูลแบบ Relational Database ที่ใช้ SQL ในการจัดการข้อมูล
เน้นการจัดเก็บข้อมูลในลักษณะของโครงสร้าง (schema-less) ซึ่งแตกต่างจากระบบฐานข้อมูลแบบ Relational Database ที่มีโครงสร้างตายตัวและต้องกำหนดโครงสร้างก่อนเก็บข้อมูล
มีความยืดหยุ่นสูง สามารถเพิ่มข้อมูลได้ง่าย ไม่จำเป็นต้องกำหนดโครงสร้างล่วงหน้า สามารถปรับเปลี่ยนโครงสร้างของข้อมูลได้ง่าย เหมาะสำหรับระบบที่มีการจัดเก็บข้อมูลที่ซับซ้อนและมีการเปลี่ยนแปลงโครงสร้างบ่อยครั้ง เช่น ระบบ Social Network ที่มีการเก็บข้อมูลผู้ใช้และโพสต์
และยังเน้นความเร็วและความปลอดภัยในการเข้าถึงข้อมูล โดยมีการจัดเก็บข้อมูลแบบ Key-Value หรือ Document ซึ่งช่วยให้การเข้าถึงข้อมูลเร็วกว่าและสามารถทำงานได้กับข้อมูลมหาศาลอย่างง่ายดาย\cite{NoSQL}

\subsection{Node.js}
NodeJS คือ Cross Platform Runtime Environment สำหรับฝั่ง Server เป็น Open Source และ Library 
ที่ใช้สำหรับพัฒนาเว็บแอปพลิเคชันต่าง ๆ ด้วยภาษา JavaScript  ถูกออกแบบมาเพื่อการสร้างแอปพลิเคชันที่ต้องการใช้ข้อมูลจำนวนมาก 
และมีการทำงานแบบ real-time ใช้ V8 Engine ที่ถูกพัฒนาโดย The Chromium Project สำหรับเพิ่มประสิทธิภาพการทำงานของภาษา JavaScript 
ร่วมกับ Web Browser ให้ดีมากขึ้น และมีการทำงานแบบSingle Process โดยมี Event-loop เข้ามาช่วยในการทำงานแบบ Asynchronous\cite{Node.js}

\subsection{Authentication and Authorization}
Authentication คือ กระบวนการยืนยันตัวตนของผู้ใช้งาน โดยมีการตรวจสอบว่าผู้ใช้งานเป็นใคร โดยมักจะใช้ username และ password เพื่อยืนยันตัวตน หรืออาจใช้วิธีอื่น ๆ เช่นการใช้งาน fingerprint หรือรูปหน้าใบผู้ใช้งาน
\enskip
Authorization คือ กระบวนการให้สิทธิ์ในการเข้าถึงและใช้งานทรัพยากรต่าง ๆ ให้แก่ผู้ใช้งาน โดยจะตรวจสอบว่าผู้ใช้งานมีสิทธิ์ในการเข้าถึงหรือไม่ โดยการตรวจสอบสิทธิ์สามารถใช้งานร่วมกับ Authentication ได้ เพื่อให้มั่นใจว่าผู้ใช้งานที่เข้าถึงทรัพยากรนั้น ๆ เป็นผู้ที่มีสิทธิ์\cite{Auth}

\section{ด้าน Frontend}

\subsection{React}
    React เป็นไลบรารี JavaScript ที่ช่วยให้ผู้พัฒนาสามารถสร้าง User Interface (UI) ได้อย่างง่ายดายและมีประสิทธิภาพ โดย React เน้นการสร้าง UI ที่มีการอัปเดตสถานะ (state) 
และการใช้งาน Component ในการแบ่งแยกหน้าที่การแสดงผลออกจากโค้ดหลัก โดย React นั้นได้รับความนิยมเนื่องจากมีความยืดหยุ่นสูง รองรับการพัฒนาแอปพลิเคชันแบบ Single Page Application (SPA) 
และสามารถใช้งานร่วมกับไลบรารีและเครื่องมืออื่น ๆ ได้อย่างคล่องตัว
    React มีโครงสร้างหลัก 3 ส่วน คือ Element, Component และ JSX โดย Element เป็นตัวแทนของ Element ใน HTML สามารถสร้าง Element ได้โดยใช้ React.createElement() 
และ Component เป็นส่วนประกอบของ UI ที่มีการจัดการข้อมูลและการแสดงผลโดยเฉพาะ สามารถสร้าง Component ด้วยการสร้าง Class หรือ Function และ JSX เป็นการเขียนโค้ดของ React 
ที่ผสมผสานระหว่าง JavaScript และ HTML ซึ่งจะถูกแปลงเป็น JavaScript โดย Babel\cite{React}

\subsection{JavaScript}
JavaScript เป็นภาษาโปรแกรมมิ่งที่ใช้กันอย่างแพร่หลายในเว็บไซต์และแอปพลิเคชันต่างๆ โดยทั่วไป JavaScript ใช้สำหรับการสร้างและจัดการเว็บไซต์ โดย JavaScript 
เป็นภาษาโปรแกรมมิ่งที่รองรับการทำงานแบบ event-driven ทำให้สามารถเชื่อมต่อกับผู้ใช้งานผ่าน browser ได้ และใช้สำหรับการสร้างฟังก์ชั่นต่างๆ บนเว็บไซต์ เช่น การตรวจสอบฟอร์ม 
การจัดการกับข้อมูลผู้ใช้ การสร้างอินเตอร์แอคทีฟที่ได้รับความนิยมเช่น React และ Angular ซึ่งทั้งสองตัวนี้เป็น JavaScript frameworks ที่ใช้สำหรับการพัฒนาเว็บแอปพลิเคชันที่ใหญ่โตและซับซ้อนได้\cite{JavaScript}

\section{ด้าน Hardware}

\subsection{Microcontroller Board}
Microcontroller Board คืออุปกรณ์อิเล็กทรอนิกส์ที่มีหน้าที่ควบคุมการทำงานของระบบหรืออุปกรณ์ต่างๆ โดยมีซอฟต์แวร์ที่รันบนไมโครคอนโทรลเลอร์เพื่อควบคุมการทำงานของอุปกรณ์ มีหลายแบบ หลายรุ่น และหลายแพลตฟอร์มเพื่อรองรับการพัฒนาและปรับปรุงระบบต่างๆ ในด้านต่างๆ เช่น ระบบควบคุมการเคลื่อนที่ในโมเดล RC Car, 
ระบบควบคุมการทำงานของหุ่นยนต์ หรืออุปกรณ์ IoT อื่นๆ ในปัจจุบันได้รับความนิยมในการนำมาใช้งานเพราะมีขนาดเล็ก การใช้พลังงานต่ำ รวมถึงความสามารถในการทำงานที่หลากหลายและน่าสนใจ ด้วยราคาที่เหมาะสมกับการใช้งานจริง ซึ่งจะมีความซับซ้อนแตกต่างกันไปขึ้นอยู่กับการใช้งานและแต่ละโมเดลของไมโครคอนโทรลเลอร์ที่ใช้งาน\cite{Micro}

\subsection{ESP32-CAM}
ESP32-CAM คือ โมดูลกล้องขนาดเล็กที่ใช้พลังงานต่ำ ใช้ชิป ESP32 มาพร้อมกับกล้อง OV2640 และมีช่องเสียบ SD Card ในตัว สามารถเชื่อมต่อ WiFi+Bluetooth เพื่อการควบคุมระยะไกลได้ 
การใช้งาน ESP32-CAM สามาถนำไปใช้ได้ตั้งแต่อุปกรณ์ IoT ธรรมดาไปจนถึงขั้นสูงอื่น ๆ สำหรับการตรวจสอบและจดจำใบหน้าโดยใช้ AI และแม้กระทั่งทำเป็นกล้องวงจรปิด


% define a command that produces some filler text, the lorem ipsum.
  % To include an image in the figure, say myimage.pdf, you could use
  % the following code. Look up the documentation for the package
  % graphicx for more information.
  % \includegraphics[width=\textwidth]{myimage}

% This code demonstrates how to get a landscape table or figure. It
% uses the package lscape to turn everything but the page number into
% landscape orientation. Everything should be included within an
% \afterpage{ .... } to avoid causing a page break too early.


\section{\ifenglish%
\ifcpe CPE \else ISNE \fi knowledge used, applied, or integrated in this project
\else%
ความรู้ตามหลักสูตรซึ่งถูกนำมาใช้หรือบูรณาการในโครงงาน
\fi
}
การสร้างเว็บแอปพลิเคชันได้มีการนำความรู้จากวิชา Basic CPE Lab ในการออกแบบเว็บแอปพลิเคชัน มีการใช้ความรู้ในวิชา Database
ในการออกแบบบฐานข้อมูล นำความรู้จากวิชา Computer Vision มาประยุกต์ใช้ในการทำ Object detection 
และมีการลงพื้นที่สำรวจความต้องการของผู้ใช้และผู้มีส่วนได้ส่วนเสียเพื่อให้โครงงานสามารถแก้ไขปัญหาได้ตรงตามกับความต้องการที่ได้รับ 
ซึ่งเป็นกระบวนการทำงานที่ได้รับจากวิชา Software Engineering และ Innovation to Market

\section{\ifenglish%
Extracurricular knowledge used, applied, or integrated in this project
\else%
ความรู้นอกหลักสูตรซึ่งถูกนำมาใช้หรือบูรณาการในโครงงาน
\fi
}
โครงงานนี้มีการใช้ความรู้ด้านการออกแบบ UI/UX มาช่วยในการออกแบบเว็บแอปพลิเคชัน และมีการศึกษาเกี่ยวกับการใช้งาน ESP32-CAM 
