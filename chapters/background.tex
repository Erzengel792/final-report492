\chapter{\ifenglish Background Knowledge and Theory\else ทฤษฎีที่เกี่ยวข้อง\fi}

การทำโครงงาน เริ่มต้นด้วยการศึกษาค้นคว้า ทฤษฎีที่เกี่ยวข้อง หรือ งานวิจัย/โครงงาน ที่เคยมีผู้นำเสนอไว้แล้ว ซึ่งเนื้อหาในบทนี้ก็จะเกี่ยวกับการอธิบายถึงสิ่งที่เกี่ยวข้องกับโครงงาน เพื่อให้ผู้อ่านเข้าใจเนื้อหาในบทถัดๆ ไปได้ง่ายขึ้น

\section{ด้าน Backend}
The text for Section 1 goes here.

\subsection{Computer Vision}

Subsection 1 text

\subsection{Machine Learning}
Subsubsection 1 text

\subsection{Database}
Subsubsection 2 text

\subsection{Alghorithm}
Subsubsection 2 text

\section{ด้าน Frontend}


\subsection{React}
    React เป็นไลบรารี JavaScript ที่ช่วยให้ผู้พัฒนาสามารถสร้าง User Interface (UI) ได้อย่างง่ายดายและมีประสิทธิภาพ โดย React เน้นการสร้าง UI ที่มีการอัปเดตสถานะ (state) 
และการใช้งาน Component ในการแบ่งแยกหน้าที่การแสดงผลออกจากโค้ดหลัก โดย React นั้นได้รับความนิยมเนื่องจากมีความยืดหยุ่นสูง รองรับการพัฒนาแอปพลิเคชันแบบ Single Page Application (SPA) 
และสามารถใช้งานร่วมกับไลบรารีและเครื่องมืออื่น ๆ ได้อย่างคล่องตัว
    React มีโครงสร้างหลัก 3 ส่วน คือ Element, Component และ JSX โดย Element เป็นตัวแทนของ Element ใน HTML สามารถสร้าง Element ได้โดยใช้ React.createElement() 
และ Component เป็นส่วนประกอบของ UI ที่มีการจัดการข้อมูลและการแสดงผลโดยเฉพาะ สามารถสร้าง Component ด้วยการสร้าง Class หรือ Function และ JSX เป็นการเขียนโค้ดของ React 
ที่ผสมผสานระหว่าง JavaScript และ HTML ซึ่งจะถูกแปลงเป็น JavaScript โดย Babel\cite{React}
% define a command that produces some filler text, the lorem ipsum.


  % To include an image in the figure, say myimage.pdf, you could use
  % the following code. Look up the documentation for the package
  % graphicx for more information.
  % \includegraphics[width=\textwidth]{myimage}

% This code demonstrates how to get a landscape table or figure. It
% uses the package lscape to turn everything but the page number into
% landscape orientation. Everything should be included within an
% \afterpage{ .... } to avoid causing a page break too early.

\section{Overfull hbox}

When the \verb.semifinal. option is passed to the \verb.cpecmu. document class,
any line that is longer than the line width, i.e., an overfull hbox, will be
highlighted with a black solid rule:
\begin{center}
\begin{minipage}{2em}
juxtaposition
\end{minipage}
\end{center}

\section{\ifenglish%
\ifcpe CPE \else ISNE \fi knowledge used, applied, or integrated in this project
\else%
ความรู้ตามหลักสูตรซึ่งถูกนำมาใช้หรือบูรณาการในโครงงาน
\fi
}

อธิบายถึงความรู้ และแนวทางการนำความรู้ต่างๆ ที่ได้เรียนตามหลักสูตร ซึ่งถูกนำมาใช้ในโครงงาน

\section{\ifenglish%
Extracurricular knowledge used, applied, or integrated in this project
\else%
ความรู้นอกหลักสูตรซึ่งถูกนำมาใช้หรือบูรณาการในโครงงาน
\fi
}

อธิบายถึงความรู้ต่างๆ ที่เรียนรู้ด้วยตนเอง และแนวทางการนำความรู้เหล่านั้นมาใช้ในโครงงาน
