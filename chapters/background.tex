\chapter{\ifenglish Background Knowledge and Theory\else ทฤษฎีที่เกี่ยวข้อง\fi}

ในบทนี้ก็จะเกี่ยวกับการอธิบายถึงสิ่งที่เกี่ยวข้องกับโครงงาน เพื่อให้ผู้อ่านเข้าใจเนื้อหาในบทถัดๆ ไปได้ง่ายขึ้น

\section{ด้าน Backend}

\subsection{Computer Vision}
Computer Vision เป็นกลุ่มของเทคโนโลยีและเทคนิคการประมวลผลภาพและวิดีโอด้วยคอมพิวเตอร์ โดยการนำเอาสัญญาณภาพและวิดีโอจากกล้องหรือเซ็นเซอร์ไปประมวลผลด้วยโปรแกรมคอมพิวเตอร์เพื่อให้เข้าใจและวิเคราะห์ภาพได้อย่างถูกต้องและมีประสิทธิภาพ
\enskip
โดย Computer Vision สามารถใช้งานได้ในหลายแขนงเช่น การตรวจจับวัตถุ (object detection) การติดตามวัตถุ (object tracking) การจำแนกวัตถุ (object recognition) การแยกแยะสิ่งของในภาพ (image segmentation) การประมวลผลภาพแบบสัมพันธ์ (image processing) และอื่นๆ\cite{Vision}

\subsection{Machine Learning}
Machine Learning คือการเรียนรู้และวิเคราะห์ข้อมูลด้วยอัลกอริทึมเพื่อทำนายและการตัดสินใจโดยไม่ต้องระบุมาก่อนว่าคำตอบจะเป็นอย่างไร
เช่น การจำแนกภาพ การสแกนเอกสาร การจัดกลุ่มหรือแนะนำสินค้า ซึ่งจะมีการใช้ข้อมูลที่มีปริมาณมากและมีความซับซ้อน 
การเรียนรู้ของ Machine Learning จะเป็นการให้คอมพิวเตอร์เรียนรู้จากข้อมูล เพื่อทำนายผลลัพธ์ที่ถูกต้องที่สุดเท่าที่เป็นไปได้ สามารถแบ่งออกเป็น 3 ประเภท ได้แก่
\begin{enumerate}
  \item Supervised Learning: เป็นการเรียนรู้ด้วยข้อมูลที่มีการระบุผลลัพธ์ให้แล้ว เช่น การจำแนกภาพว่าเป็นแมวหรือหมา
  \item Unsupervised Learning: เป็นการเรียนรู้โดยไม่มีการระบุผลลัพธ์ให้แล้ว แต่จะต้องหาลักษณะหรือลำดับที่เป็นลักษณะเด่นเพื่อแบ่งกลุ่มข้อมูล เช่น การจัดกลุ่มลูกค้าที่มีพฤติกรรมการซื้อสินค้าคล้ายกัน
  \item Reinforcement Learning: เป็นการเรียนรู้โดยมีการให้รางวัลหรือลบคะแนนเมื่อทำกิจกรรมใดๆ เช่น การเล่นเกม
\end{enumerate}
\cite{Machine}
\subsection{NoSQL Database}
NoSQL ย่อมาจาก "Not Only SQL" ซึ่งเป็นระบบฐานข้อมูลที่ไม่ใช้ภาษา SQL ในการจัดการข้อมูล ต่างจากระบบฐานข้อมูลแบบ Relational Database ที่ใช้ SQL ในการจัดการข้อมูล
เน้นการจัดเก็บข้อมูลในลักษณะของโครงสร้าง (schema-less) ซึ่งแตกต่างจากระบบฐานข้อมูลแบบ Relational Database ที่มีโครงสร้างตายตัวและต้องกำหนดโครงสร้างก่อนเก็บข้อมูล
มีความยืดหยุ่นสูง สามารถเพิ่มข้อมูลได้ง่าย ไม่จำเป็นต้องกำหนดโครงสร้างล่วงหน้า สามารถปรับเปลี่ยนโครงสร้างของข้อมูลได้ง่าย เหมาะสำหรับระบบที่มีการจัดเก็บข้อมูลที่ซับซ้อนและมีการเปลี่ยนแปลงโครงสร้างบ่อยครั้ง เช่น ระบบ Social Network ที่มีการเก็บข้อมูลผู้ใช้และโพสต์
และยังเน้นความเร็วและความปลอดภัยในการเข้าถึงข้อมูล โดยมีการจัดเก็บข้อมูลแบบ Key-Value หรือ Document ซึ่งช่วยให้การเข้าถึงข้อมูลเร็วกว่าและสามารถทำงานได้กับข้อมูลมหาศาลอย่างง่ายดาย\cite{NoSQL}

\subsection{AWS Rekognition}
AWS Rekognition เป็นบริการในคลาวด์ที่มีความสามารถในการรู้จำและวิเคราะห์ภาพและวิดีโออัตโนมัติ สามารถจับและระบุวัตถุที่ปรากฏในรูปภาพหรือวิดีโอ โดยรวมถึงการจัดหมวดหมู่และระบุตำแหน่งของวัตถุ

\subsection{AWS S3}
AWS S3 หรือ Simple Storage Service เป็นบริการจัดเก็บข้อมูลแบบคลาวด์ที่ใช้ในโครงการ เพื่อจัดเก็บข้อมูลสำคัญ เช่น ไฟล์สื่อ, รูปภาพ, และเอกสาร การใช้ AWS S3 ช่วยให้สามารถเก็บข้อมูลอย่างปลอดภัยและมีความยืดหยุ่นในการเข้าถึงข้อมูลจากที่ใดก็ได้

\subsection{AWS Lambda}
บริการ AWS Lambda เป็นบริการที่ใช้สำหรับรันโค้ดโดยอัตโนมัติตามเหตุการณ์ที่เกิดขึ้น โดยไม่ต้องการการจัดการเซิร์ฟเวอร์เอง เราได้ใช้บริการ AWS Lambda เพื่อสร้างฟังก์ชันที่ทำงานอัตโนมัติเมื่อเหตุการณ์เกิดขึ้น เช่นการประมวลผลข้อมูลหลายๆอย่าง อย่างรวดเร็ว การใช้บริการนี้ช่วยให้การจัดการและประมวลผลข้อมูลเป็นไปอย่างมีประสิทธิภาพและเหมาะสมกับความต้องการของโครงการ

\subsection{AWS DynamoDB}
เป็นบริการฐานข้อมูล NoSQL ที่ใช้โครงสร้างข้อมูลแบบอนุกรม (schema-less) ซึ่งไม่จำเป็นต้องกำหนดโครงสร้างของข้อมูลล่วงหน้า แต่สามารถเพิ่มและเปลี่ยนแปลงข้อมูลในฐานข้อมูลได้ตามต้องการ สามารถใช้ร่วมกับบริการ AWS อื่น ๆ เช่น AWS Lambda เพื่อสร้างแอปพลิเคชันที่รับเหตุการณ์และประมวลผลข้อมูลอัตโนมัติ

\subsection{AWS API Gateway}
เป็นบริการที่ใช้ในการสร้าง, จัดการ, และเรียกใช้ Application Programming Interfaces (APIs) สามารถใช้ร่วมกับบริการอื่น ๆ ของ AWS เช่น AWS Lambda, AWS Cognito, AWS S3, เป็นต้น เพื่อสร้างและจัดการ APIs ที่มีความยืดหยุนและความสามารถในการปรับปรุงด้านพื้นฐาน

\subsection{Arduino IDE}
Arduino IDE เป็นเครื่องมือสำหรับการพัฒนาซอฟต์แวร์ ถูกออกแบบโดยเฉพาะสำหรับการพัฒนาโปรแกรมสำหรับบอร์ด Arduino ซึ่งเป็นแพลตฟอร์มฮาร์ดแวร์ที่ใช้ในการสร้างและควบคุมอุปกรณ์อิเล็กทรอนิกส์ต่าง ๆ

% สำหรับฝั่ง Server เป็น Open Source และ Library 
% ที่ใช้สำหรับพัฒนาเว็บแอปพลิเคชันต่าง ๆ ด้วยภาษา JavaScript  ถูกออกแบบมาเพื่อการสร้างแอปพลิเคชันที่ต้องการใช้ข้อมูลจำนวนมาก 
% และมีการทำงานแบบ real-time ใช้ V8 Engine ที่ถูกพัฒนาโดย The Chromium Project สำหรับเพิ่มประสิทธิภาพการทำงานของภาษา JavaScript 
% ร่วมกับ Web Browser ให้ดีมากขึ้น และมีการทำงานแบบSingle Process โดยมี Event-loop เข้ามาช่วยในการทำงานแบบ Asynchronous\cite{Node.js}

% \subsection{Authentication and Authorization}
% Authentication คือ กระบวนการยืนยันตัวตนของผู้ใช้งาน โดยมีการตรวจสอบว่าผู้ใช้งานเป็นใคร โดยมักจะใช้ username และ password เพื่อยืนยันตัวตน หรืออาจใช้วิธีอื่น ๆ เช่นการใช้งาน fingerprint หรือรูปหน้าใบผู้ใช้งาน
% \enskip
% Authorization คือ กระบวนการให้สิทธิ์ในการเข้าถึงและใช้งานทรัพยากรต่าง ๆ ให้แก่ผู้ใช้งาน โดยจะตรวจสอบว่าผู้ใช้งานมีสิทธิ์ในการเข้าถึงหรือไม่ โดยการตรวจสอบสิทธิ์สามารถใช้งานร่วมกับ Authentication ได้ เพื่อให้มั่นใจว่าผู้ใช้งานที่เข้าถึงทรัพยากรนั้น ๆ เป็นผู้ที่มีสิทธิ์\cite{Auth}

\section{ด้าน Frontend}

\subsection{Next.js}
Next.js คือ open-source React framework เป็นการใช้ Server Side Rendering (SSR) และยังสามารถทำเว็ปไซต์ได้ทั้งแบบ static และ dynamic 
\enskip
ซึ่งข้อดีของการเป็น Server Side Rendering (SSR) คือ ช่วยในเรื่อง Search Engine Optimization (SEO) เพราะถ้าทำการ inspect เว็ปไซต์ที่สร้างโดย Next.js จะพบว่า source เป็น html ฆส่วนใหญ่ ซึ่งทำให้ SEO ค้นผ่าน source เพื่อให้ได้ข้อมูลและจัดหมวดหมู่ได้ง่ายกว่า React ที่เป็น Javascript มากกว่า

\subsection{Vercel}
Vercel เป็นบริการโฮสติ้งและการจัดการเว็บแอปพลิเคชันแบบคลาวด์ที่มีความยืดหยุ่นและสามารถดำเนินการได้รวดเร็ว รองรับหลายภาษาและเทคโนโลยีในการพัฒนา เช่น JavaScript, Node.js, React และ มีความสามารถในการปรับปรุงแอปพลิเคชันอัตโนมัติ

\section{ด้าน Hardware}

\subsection{Microcontroller Board}
Microcontroller Board คืออุปกรณ์อิเล็กทรอนิกส์ที่มีหน้าที่ควบคุมการทำงานของระบบหรืออุปกรณ์ต่างๆ โดยมีซอฟต์แวร์ที่รันบนไมโครคอนโทรลเลอร์เพื่อควบคุมการทำงานของอุปกรณ์ มีหลายแบบ หลายรุ่น และหลายแพลตฟอร์มเพื่อรองรับการพัฒนาและปรับปรุงระบบต่างๆ ในด้านต่างๆ เช่น ระบบควบคุมการเคลื่อนที่ในโมเดล RC Car, 
ระบบควบคุมการทำงานของหุ่นยนต์ หรืออุปกรณ์ IoT อื่นๆ ในปัจจุบันได้รับความนิยมในการนำมาใช้งานเพราะมีขนาดเล็ก การใช้พลังงานต่ำ รวมถึงความสามารถในการทำงานที่หลากหลายและน่าสนใจ ด้วยราคาที่เหมาะสมกับการใช้งานจริง ซึ่งจะมีความซับซ้อนแตกต่างกันไปขึ้นอยู่กับการใช้งานและแต่ละโมเดลของไมโครคอนโทรลเลอร์ที่ใช้งาน\cite{Micro}

\subsection{ESP32-CAM}
ESP32-CAM คือ โมดูลกล้องขนาดเล็กที่ใช้พลังงานต่ำ ใช้ชิป ESP32 มาพร้อมกับกล้อง OV2640 และมีช่องเสียบ SD Card ในตัว สามารถเชื่อมต่อ WiFi+Bluetooth เพื่อการควบคุมระยะไกลได้ 
การใช้งาน ESP32-CAM สามาถนำไปใช้ได้ตั้งแต่อุปกรณ์ IoT ธรรมดาไปจนถึงขั้นสูงอื่น ๆ สำหรับการตรวจสอบและจดจำใบหน้าโดยใช้ AI และแม้กระทั่งทำเป็นกล้องวงจรปิด


% define a command that produces some filler text, the lorem ipsum.
  % To include an image in the figure, say myimage.pdf, you could use
  % the following code. Look up the documentation for the package
  % graphicx for more information.
  % \includegraphics[width=\textwidth]{myimage}

% This code demonstrates how to get a landscape table or figure. It
% uses the package lscape to turn everything but the page number into
% landscape orientation. Everything should be included within an
% \afterpage{ .... } to avoid causing a page break too early.


\section{\ifenglish%
\ifcpe CPE \else ISNE \fi knowledge used, applied, or integrated in this project
\else%
ความรู้ตามหลักสูตรซึ่งถูกนำมาใช้หรือบูรณาการในโครงงาน
\fi
}
การสร้างเว็บแอปพลิเคชันได้มีการนำความรู้จากวิชา Basic CPE Lab ในการออกแบบเว็บแอปพลิเคชัน มีการใช้ความรู้ในวิชา Fundamentals Database
ในการออกแบบบฐานข้อมูล นำความรู้จากวิชา Computer Vision มาประยุกต์ใช้ในการทำ Object detection 
และมีการลงพื้นที่สำรวจความต้องการของผู้ใช้และผู้มีส่วนได้ส่วนเสียเพื่อให้โครงงานสามารถแก้ไขปัญหาได้ตรงตามกับความต้องการที่ได้รับ 
ซึ่งเป็นกระบวนการทำงานที่ได้รับจากวิชา Software Engineering และ Innovation to Market

\section{\ifenglish%
Extracurricular knowledge used, applied, or integrated in this project
\else%
ความรู้นอกหลักสูตรซึ่งถูกนำมาใช้หรือบูรณาการในโครงงาน
\fi
}
โครงงานนี้มีการใช้ความรู้ด้านการออกแบบ UI/UX มาช่วยในการออกแบบเว็บแอปพลิเคชัน การใช้งานเทคโนโลยีคลาวด์ ความรู้ในการใช้งานและดูแลระบบคลาวด์เป็นสิ่งสำคัญเพื่อความเสถียรและปลอดภัยของระบบ และมีการศึกษาเกี่ยวกับการใช้งาน ESP32-CAM 
